\documentclass[11pt]{article}
\usepackage{amsmath, amssymb}
\usepackage[margin=1in]{geometry}
\usepackage{times}

\title{\textbf{Concept for Summer Research Focus: \\ An Adaptive AI Framework for Ransomware Defense}}
\author{
    Alex Towell \\
    atowell@siue.edu \\
    Department of Computer Science \\
    Southern Illinois University-Edwardsville \\
}
\date{May 10, 2025}

\begin{document}

\maketitle

I've been developing some ideas for an advanced AI framework to tackle ransomware, and I'm excited about its potential for our research. I wanted to outline the core concepts and propose exploring this further as a 3-credit independent study or focused research course this summer. This could be a great way to make substantial progress and also align with broader interests, perhaps even involving Dr. Gultepe.

\section*{The Big Idea: Intelligent, Adaptive Ransomware Defense}

The core concept is an AI system that learns to proactively and intelligently handle ransomware threats. Instead of static defenses, this system would feature:

\begin{itemize}
    \item \textbf{A Cognitive Core}: An RL-trained policy (e.g., based on a DNN) that learns optimal strategies for detection, response, and mitigation.
    \item \textbf{Deep Contextual Understanding}: The policy operates on rich context (from logs, network activity, etc.) represented as latent embeddings. This allows it to discover and act on subtle patterns.
    \item \textbf{An LLM as a Dynamic Tool}: This is a key part. The RL policy itself would learn to "prompt engineer" a Large Language Model. The LLM would be used by the policy to:
        \begin{itemize}
            \item Dynamically rewrite and augment the context (e.g., pulling salient examples from an associative memory).
            \item Analyze the current environment to generate useful metadata, enriching the policy's state awareness.
        \end{itemize}
\end{itemize}

The emphasis is on a system where the policy doesn't just act, but actively learns how to best gather and process information using the LLM as a powerful, adaptable assistant. We can also incorporate non-latent features or DBN-style variables where useful, but the primary thrust is leveraging learned latent representations.

\section*{Promising Extensions to Explore}

Two particularly exciting avenues we could investigate as part of this research include:

\begin{itemize}
    \item \textbf{Ransomware-Specific LLM}: Pretraining/fine-tuning the LLM on cybersecurity data to make it a "ransomware expert," significantly boosting its effectiveness within the framework.
    \item \textbf{Adversarial Training}: Developing the defense policy by pitting it against an AI-driven attacker. This would push the defender to become exceptionally robust and adaptive to novel threats.
\end{itemize}

\section*{Summer Research Focus (3-Credit Project/Course)}

I propose dedicating the summer (as a 3-credit independent study, directed research, or a custom online course module) to formally dive into this. The main goals would be:
\begin{itemize}
    \item To further develop the theoretical underpinnings of this framework.
    \item Potentially prototype key components (e.g., the policy-LLM interaction loop or the context embedding strategy).
    \item Lay the groundwork for future publications.
\end{itemize}
Given my self-driven nature, I'm confident I can make significant headway. This would directly contribute to my dissertation research and our group's objectives.

\section*{Next Steps}

I'd love to discuss this with you further at your convenience to see if this aligns with your vision for my summer research and how we might best structure it.

\end{document}